\chapter{実装}

本章では,提案するシステムの具体的な実装について述べる.
まず使用した機材や技術情報などについて明確にし,次に提案するシステムの流れを全体像とともに説明する.

\section{音声合成ソフト}
本研究では音声合成ソフトウエアとしてオープンソフトのOpen JTalk\cite{jtalk}を用いる.Open JTalk は形態素解析部に MeCab\cite{mecab},発音辞書に NAIST Japanese Dictienary\cite{dic},波形生成部にhts−engine API\cite{hmm}を組み込んでいる.感情パラメタとして,NOMAL,HAPPY,ANGRY,BASHFUL,SADの5種類が用意されている.

\section{書籍データ}
青空文庫\cite{aozora}とは著作権が消滅した作品や著者が許諾した作品のテキストを公開しているインターネット上の電子図書館である.この中から年代に依存しない現代的な話し言葉が用いられている童謡を中心に30作品\(予定\)を書籍データとして用いる.なお,ルビのデータが含まれているため予めタグを除いておく.

\section{開発環境}
\section{まとめ}
本章では,提案するシステムを実装するにあたり利用したハードウェア及びソフトウェア情報について述べ,具体的な手法の実装部分の詳細について述べた.
