\chapter{関連研究}
本章では音声合成や朗読システム感情推定に関する既存の研究について紹介し,それに対しての本研究の位置づけと意義を述べる.

音声合成による朗読システムとして吉田ら\cite{yoshida}は、文間ポーズ長の頻度分布調査実験の結果を基に文内・文末表層情報に着目している.
文内情報\(命令,否定,意志等\)と文末情報\(「〜ある」,「〜いる」,「〜んだ」等\)をカテゴリーに分け,文間ポーズと基本周波数,話速を調整している.

文章の意味内容に着目した研究では,布目ら\cite{fume}はポーズ情報の推定と感情表現の推定を行っている.ポーズ推定では,タイトルや章立て構造といった文章の論理要素に応じてポーズ長を推定し,ポーズを挿入する.感情推定では「喜び」「怒り」「悲しみ」及び「平静」の各感情の推定モデルを作成しスコアを算出し最もスコアの高い感情を文にラベリングする.そのラベリングに応じて,文ごとに韻律辞書や音声制御用パラメタを切り替えて読み上げる.
この研究では,意味内容に着目して感情の各スコアを算出しているが最終的にラベリングを行ってしまっているので,各感情の中間の表現ができ
ない.また,各単語に分けてラベルとの類似度を算出してしまっているので,例えば文末の「でない」といった否定によって文全体の文意が反転する場合に対応できていない可能性がある.

一方で,自然言語の感情分析の分野では,Hongら\cite{hong}は単語の繋がりを考慮して文全体のポジティブ度ネガポジ度を判定している.また,IBMのTone Analyzerでは各文の感情\(Anger, Disgust, Fearm, Joy, Sadnessなど\)とその度合を推定することができる.
