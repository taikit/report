\chapter{関連研究}
%TODO 最後にまとめ

本章では,音声合成や朗読システムにの既存研究について述べるとともに本研究の明確な位置づけを行う.

\section{音声合成の研究}\label{research:speech_synthesis}
感情表現が可能な音声合成技術の研究を紹介する.

\subsection{様々な感情表現が可能な音声合成手法}
近年,日本語の音声合成手法として広く利用されている2002年のYoshimura\cite{yoshimura}が提案した隠れマルコフモデル(HMM)に基づく合成手法がある.
この手法以前もカーナビや音声翻訳システムなどに音声合成が利用されていた.
しかし,様々な話者の声質で話したり,嬉しそうに,怒ったようになど様々な発話スタイルで話すことができるものは少なかった.
なぜなら,それ以前の波形接続型のシステムで様々な声質,発話スタイルを実現するためには,様々な声質,発話スタイルで収録された膨大な量の音声データを処理しなければならず,また合成するには膨大な量の素片を格納する記憶媒体が必要となるため,実現は非常に困難であったためである.
このHMMに基づく合成手法を用いることで,抽象化された関数を利用し音声波形に揺らぎあってもあってもその背後に潜んでいる特徴を見出すことができる.
これによって少ない音声データで学習することが可能になり,パラメータ調整することで別の人物の声を真似たり,様々な感情表現を行うことが可能になった.

\subsection{直接波形接続型音声合成における感情表現}
飯田ら\cite{iida}は自然音声直接波形接続型音声合成システムCHATERを用いて,表現したい感情ごと(喜び・怒り・悲しみ)にその感情に対応する音声コーパスから最適な音声波形素片を選択し接続することで,音声を合成した.
聴取実験を行い,有意水準1\%で検定を行ったところ,各感情は有意に判別され,各感情は有意に判別された.
しかし,波形接続型の音声合成では合成したい音素の基本周波数予測エラーによりイントネーションが不自然に聞こえる場合があり,そもそもの音声合成の質が高いとは言えない.
また,前小節で述べた通り波形接続型の音声合成では学習に多くの音声データが必要であり,様々な感情のデータを取得するには多くのコストがかかる

\subsection{HMM型音声合成における感情表現}
都築ら\cite{tsuduki}はHMM音声合成システムを用いて少ない学習データを用いて感情表現のモデル化を行っている.
学習に用いる音声データは英語音声で目的の感情を表現しやすい文章を読み上げたものである.
感情は平静・怒り・喜び・悲しみの4種類の感情について合成を行い,聴取による主観評価実験を行っている.
結果としては,判別誤りが多く見られ芳しい結果は得られなかった.
原因として文章と感情の関係や音声品質,学習データの不足などが挙げられている.
%TODO 本研究との関係は

\subsection{HMM型音声合成における共有感情付与モデルを利用した感情表現}
大谷\cite{otani}らはHMM型音声合成において加算構造に基づく感情表現を提案している.
この手法では複数の話者の感情音声データ用いて,学習者共有の感情成分を持つ共有感情付与モデルを学習し,このモデルを任意の平静音声モデルへ適応する.
これによりHMM型音声合成において従来法に比べ,音質が高く,安定した感情表現が可能となった.

%TODO 推定の関連研究は?
%\section{感情推定の研究}

%TODO 朗読の研究

\section{朗読システムの研究}
物語の文章内容に応じて,音声合成を調整する音声合成システムの研究について説明する.

\subsection{表層情報を用いた朗読システム}
吉田ら\cite{yoshida}は朗読文に朗読者の音声の間(ポーズ)及び韻律的特徴(基本周波数,話速)を解析し特徴のモデル化を行うことで,文章に応じてポーズや韻律を付与するシステムを提案している.
物語の分は文内表層情報(命令,否定,意志等)と文末表層情報(「〜ある」,「〜いる」,「〜んだ」等)よりカテゴリー分けし,それを特徴としている.
聴取実験の結果では調整を行っていない音声に比べ調整を行った音声の方が自然と回答した割合が80\%前後であった.
しかし,この研究では予め「情景描写」や「緊迫」といった場面抽出を人手で行っているため,それ以外の場面において有効であるか疑問である.
また,基本周波数と話速の調節しか行っておらず,前節で述べた感情付与モデルを利用して音声合成を調節することで,さらに自然になる可能性がある.

\subsection{感情推定を用いた朗読システム}
布目ら\cite{fume}はセリフ文に対し,文中やその隣接分に出現する表記や単語を手がかりにして,事前に定義された複数の感情から最も近い感情表現を割り当てるシステムを提案している.
感情推定では「喜び」「怒り」「悲しみ」及び「平静」の各感情を付与した学習データを作成する.
ナイーブベイズを用いて学習を行い,推定ではスコアを算出し最もスコアの高い感情を文に付与する.
その推定をもとに,文ごとに韻律辞書や音声制御用パラメタを切り替えて読み上げる.
精度評価の結果,喜,怒,哀の3種の感情ラベルに関しては90%前後の精度を得た.
しかしながら,感情を付与する対象はセリフのみであり,セリフ文以外にも付与することでより自然な朗読が可能になると考えられる.
また,感情ラベルの付け方が明示されておらず,推定が容易な文のみを対象としていたり客観的なラベル付けが行われていない可能性がある.


\section{本研究の位置づけ}
本研究では自然な朗読システムのために物語の全文を対象に一文ずつ予め用意された感情の中からで音声合成すべきか推定を行う.


大谷\cite{otani}らの研究ではすでに感情表現が可能な質の高い音声合成は可能であることがわかった.
しかし,この技術だけでは人でによる感情のパラメタ指定が必要であり,コストがかかる.
本研究で期待される感情推定技術と組み合わせることで,自然で質の高い自動朗読システムを実現することが可能になる.

本研究と同じ目的の研究は他にもあり,それらとの違いや類似点を説明する.


吉田ら\cite{yoshida}の研究では基本周波数と話速の調整のみを行っていたが,本研究ではあくまで感情ラベルの推定を行う.
これによって感情を考慮した朗読システムが構築できる可能性がある.
ただし,この研究によって文脈や内容そのものを理解せずとも理解せずとも,文の表層情報からある程度,どのように朗読すべきか推定可能性が示された.
これを受けて,本研究では機能語のみによる推定も行った.


布目ら\cite{fume}はセリフ文のみに着目している.
本研究ではセリフ文以外の文章に対して感情推定を行う重要性の検討を行い,それに対する推定も行う.
また,ナイーブベイズのみの推定になっているが,本研究ではその他にランダムフォレストやSVMでの比較実験を行い評価する.

さらに既存の研究では,正解ラベルの付け方,使用する文の選定に不明瞭な点が多く,実際の運用の際に未知の文に対応できるのか疑問であった.
本研究では,選択した物語からすべての文を対象に無作為に抽出を行い,複数の第三者によって評価を行わせ正解データを作成した.


\section{本章のまとめ}
本章では感情表現が可能な音声合成の研究と朗読システムの研究をいくつか挙げた.
また,音声合成技術と本研究を組み合わせることで得られるメリットを説明した.
さらに既存研究の現状とそれらが抱える問題点を指摘した.
それを踏まえ本研究が目指す領域について説明した.
