\chapter{関連研究}
本研究に関連する過去の研究について述べる.

感情を考慮した音声合成の研究として大谷\cite{otani}らは共有感情モデルを構築してる.
このモデルを用いることで音声を合成する際に感情を表出させることが可能である.
しかし,感情パラメタ自体は人手で与える必要がある.
本研究と組み合わせることで自動的に感情を推定しパラメタを与えることで,文そのもののみで感情豊かな音声合成が可能となる.

吉田ら\cite{yoshida}は自然な朗読システムのために文内・文末表層情報に着目している.
文内情報(命令,否定,意志等)と文末情報(「〜ある」,「〜いる」,「〜んだ」等)をカテゴリー分けし,実際の朗読音声を元に文間ポーズと基本周波数,話速のモデル化を行うことで推定を実現している.
しかし,この研究はあくまで自然な朗読システムの構築を目的としてるため,感情が考慮されてるとは言えない.
本研究では物語から喜怒哀楽といった感情を推定することで,感情豊かな朗読システムの実現を目指している.

布目ら\cite{fume}はポーズ情報の推定と感情表現の推定を行っている.
ポーズ推定では,タイトルや章立て構造といった文章の論理要素に応じてポーズ長を推定し,ポーズを挿入する.
感情推定では「喜び」「怒り」「悲しみ」及び「平静」の各感情を付与した学習データを作成する.
ナイーブベイズを用いて学習を行い,推定ではスコアを算出し最もスコアの高い感情を文に付与する.
その推定をもとに,文ごとに韻律辞書や音声制御用パラメタを切り替えて読み上げる.
しかしながら,感情を付与する対象はセリフのみであり,本研究ではセリフ以外の文についても感情を付与する.
