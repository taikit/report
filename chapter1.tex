\begin{comment}
- 主張
オーディオブックの作成にあたり
- 朗読システム
  - オーディオブック
    - 市場の拡大
  - 電子書籍
  - 課題
- 音声合成
  - 感情を込められる
  - 人手によるパラメタ調整
- 感情推定
  - EmotionML
\end{comment}

\chapter{序論}

%序論ではまず,本論文の背景としてオーディオブックと音声合成について解説する.


\section{背景}

%\subsection{音声合成}
%音声合成とは,人間の音声を人工的に作り出すことである.音声合成技術は文字を読むことが困難な障害者,外国人や幼児などに画面読み上げソフトとして長く利用されてきており,言葉を発することが困難な人が代替手段として利用することも多い.さらに,21世紀に入ってからは家電製品の音声ガイダンスや公共交通機関のアナウンス,ロボットの発話用途などとして広く使用されるようになっている.
%
%
%\subsection{オーディオブック}
%オーディオブックとは主に書籍を朗読したものを録音した音声コンテツのことである.
%アメリカを中心に市場規模が拡大している.
%もともと車社会のアメリカなどの国では早期から市場が確立していたが,近年インターネットを介して気軽にダウンロードして楽しめる環境が整ったことなどによりアメリカとカナダの市場規模が2015年には前年比21\%拡大している.
%また日本においても定額配信サービスが開始されており,今後さらに普及する可能性がある.しかしながら,このようなオーディオブックは書籍から音声化する際に手間やコストが電子書籍にくらべて10倍ほどかかっており2〜3ヶ月ほどかかると言われいる.
%
%
%\section{本論文の目的}
%音声合成技術を用いて人手で行っている朗読作業を根源的な目的である.これまでの音声合成研究の結果,単に情報を伝達する目的では十分な音質が確立されている.しかし,従来の音声合成は一文やフレーズの読み上げでは高品質な音声を実現している一方で書籍データのような長い文章では平板で淡々とした読み上げになってしまい,感情的あるいは情緒的な表現を多く含む小説などの朗読を聞くには不十分である.
%近年になって, 喜怒哀楽といった感情の種類をパラメタとして与え表現豊かな音声を合成できるソフトが販売されている.しかし,これらのパラメタは文または単語ごとに人手で設定する必要がある.短い文章など限られた場合は容易であるが, 小説といった膨大な文章に対して都度人手でパラメタ調整を行うのは大変手間がかかる.
%そこで,本研究では文章から読み上がる感情として最適なものを推測することを目的とする.


%社会的背景
近年,従来の書籍の他に電子書籍など様々な書籍の楽しみ方が広がっている.
その中で,書籍の朗読は専門のナレーターによる朗読音声を収録したオーディオブックが知られている.
オーディオブックはアメリカを中心に市場規模が拡大している.
もともと車社会のアメリカなどの国では早期から市場が確立していたが,近年インターネットを介して気軽にダウンロードして楽しめる環境が整ったことなどによりアメリカとカナダの市場規模が2015年には前年比で約21\%拡大している\cite{wsj}.
日本においても定額配信サービスが開始されており,今後さらに普及する可能性がある.


しかし,このようなオーディオブックは書籍から音声化する際には手間やコストが電子書籍にくらべて10倍ほどかかっており2〜3ヶ月ほどかかると言われている.\cite{ueda}

%技術的・研究的背景・問題
そこで,電子書籍から人間の音声を人工的に作り出す音声合成技術を用いて機械で自動的に朗読するシステムの研究が行われている.
近年の音声合成技術を用いれば喜怒哀楽といった感情を指定することで感情豊かな音声を合成できる.
しかし,これらのパラメタは文もしくは単語ごとに人手で設定する必要がある.
短い文章など限られた場合は容易であるが,小説といった膨大な文章に対して都度人手でパラメタ調整を行うのは手間がかかる.

%目的
そこで本研究では未知の文に対しその文を読み上げるときの感情として最適なものを推測することを目的とする.
これにより自然な朗読システムが可能が実現可能になり,人手での手間やコストをかけずにオーディオブックを作成できるようになることが期待される.

%目的を達成するための手法
本研究では,文にどのような単語が含まれているかという出現情報をもとに機械学習技術を用いて感情を推定する.
名詞,動詞,形容,形容動詞は内容語とよばれるが物語に依存する可能性が高いため,内容を除いた機能語を用いることで内容に依存しない分類器を生成できる.
このため内容語を無視して機能語のみを用いて学習し感情の推定を行う.
セリフのみに限らずすべての文を対象に,出現情報から機能語に絞った感情推定を行っている研究は筆者の知る限り存在しない.

%実験手法の正しさの確認
本研究では,先行研究と同様にNormal,Happy,Sad,Angryの4つに感情をクラス分けする.
本手法の正しさを確認するための実験として,まず一つの文にそれぞれ4つの感情で音声を合成する.
そしてWebのアンケートシステムを用いて,被験者にこれらの音声を実際に聞いてもらい,その文を読み上げる際にどの感情が最も適しているか判定してもらう.
このように生成された学習データを用い交差検証を行うことで本手法の分類性能を評価する.

%\section{本論文の構成}
%本章では,本論文の背景となる。。。について説明した.そのうえで,。。。のニーズと課題につ いて説明し,それを踏まえ。。。。概要に触れた後,本論 文の目的を述べた.第二章では既存の関連する研究について述べる.また第三章では本論文が提案する手法の詳細を述べ,第四章でその具体的な実装につい て説明する.そのうえで第五章で提案手法の効果を測定するために行なった実験の結果と考察を述べ,六章でその結論と今後の展望について述べる.
