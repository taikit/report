\chapter{実験}

形態素解析部にMeCab\cite{mecab}

\section{実験環境}

\section{比較実験}
\subsection{セリフ文}
\subsection{分類器}
\section{評価実験}
\subsection{評価指標}
\subsection{交差検証}
%TODO 用いたグリッドサーチの候補値
%\subsection{グリッドサーチ}



\subsection{評価}
本実験では,leave-one-out交差検証を行い,判定結果に対応する入力データの集合をTP,FP,TN,FNを次のように定義する.

\begin{description}
   \item[True Positive(TP)] 実際の感情のものを実際の感情であると予測したものの件数
   \item[True Negative(TN)] 実際の感情でないものをその感情でないと予測したものの件数
   \item[False Positive(FP)] 実際の感情でないものを実際の感情であると予測したものの件数
   \item[False Negative(FN)] 実際の感情のものを実際の感情でないと予測したものの件数
 \end{description}


以上をふまえ,分類器の性能評価を式(1),(2),(3),(4)で行う.
本研究は分類推定を目的としているため特にF値(4)に注目する.

\begin{eqnarray}
  正解率(Ac) =  \frac{TP + TN} {TP + TF + NP + NF}
\end{eqnarray}

\begin{eqnarray}
  適合率(Pr) = \frac{TP} {TP + FP}
\end{eqnarray}

\begin{eqnarray}
  再現率(Re) =  \frac{TP} {TP + FN}
\end{eqnarray}

\begin{eqnarray}
  F値 =  \frac{2*Pr*Re}{Pr + Re}
\end{eqnarray}


\subsection{比較実験}
提案手法の有効性を検証するために,同様な実験をセリフ文のみに絞った場合とさらに機能語に絞らなかった場合とそれぞれ行った.
さらに,ランダムフォレストの比較としてSVMを用いた実験も行った.
このとき,ランダムフォレストと同様にSVMでもグリットサーチで最適なパラメタを導出した.
