\chapter{考察と今後の課題}

全体としてのF値は高くない結果となった.
原因として学習データが少ないことや出現を示すベクトルの形式に問題がある可能性がある.
また,グリットサーチを正確度を基準に行ってしまったためF値を基準にやり直す必要がある.

ランダムフォレストとSVMを比較する.
SVMでは,全文に機能語を絞らずに分類を行った場合を除く他のすべての場合で,推定が一つの感情に偏ってしまった.
したがって,本研究の目的のためにはSVMよりランダムフォレストの方が有用であると言える.

機能語に絞った場合ととそうでない場合を比較する.
ランダムフォレストの値ではほぼ同じもしくは機能語に絞らない方がわずかに良い結果が得られている.
これは,学習データに用いた物語が5つと少ないことやleave-one-outを用いたことで推定する文と同じ物語の文を用いて学習を行っているからであると考えられる.
したがって,機能語だけでも感情の推定を行える可能性はまだある.
実際の運用では未知の物語の文に対して推定を行うので,機能語だけの学習・推定の方が精度が高い推定が行えるかもしれない.
この検証を行うためには物語数を増やし学習データを増やした上で,leave-one-outではなく一つの物語をテストデータして他の物語を学習データとして検証を行う必要がある.
また,決定木を用いて各単語の重要度を算出することで,機能語が感情推定にどれほど寄与するのか確認することができる.
