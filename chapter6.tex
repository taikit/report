\chapter{結論}

本研究では未知の文に対しその文を読み上げるときの感情として最適なものを推測することを目的とした.
このための手法として物語に依存しがちな内容語を除いて機能語のみを用いてランダムフォレストで学習・推定する手法を提案した.


実験はネット上の5つの物語を使用して音声データを作成しWebのアンケートシステムを用いてNormal,Happy,Sad,Angryの4つのに分類し学習データを作成した.
また,比較実験として機能語のみで学習・推定するか否かやランダムフォレストの他にSVMでの実験やセリフ文のみに絞った場合を行った.


結果とした全体的に高い精度を得ることはできなかった.
しかし,本研究にはSVMよりランダムフォレストが有用であることや内容語を取り去って機能語のみで学習・分類を行っても,精度に大差はないことがわかった.
したがって,ランダムフォレストを用いて,物語を増やし学習データを増やして学習を行い未知の物語に対して推定する検証を行うことで本手法の有用性が証明される可能性がある.

