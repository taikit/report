\chapter{結果と考察}
本章では,グリッドサーチの結果と各比較実験の結果を示し、それを踏まえて考察を行う。
\section{結果}

\subsection{グリッドサーチ}
\begin{table}[ht]
 \centering
  \caption{ランダムフォレストのグリットサーチの結果}
  \vspace{0.3\baselineskip}
     \scalebox{0.75}{
  \begin{tabular}{|c|c|c|c|c|} \hline
  パラメタ & 全文 & \shortstack{全文\\(機能語のみ)} & セリフ & \shortstack{セリフ\\(機能語のみ)}\\ \hline \hline
  ceriterion & entropy & entropy & entropy & entropy\\ \hline
  min\_samples\_leaf & 10 & 12 & 2 & 2\\ \hline
  n\_estimators & 100 & 50 & 30 & 10\\ \hline
  max\_features & None & None & sqrt & sqrt\\ \hline
  min\_samples\_split & 5 & 5 & 2 & 5\\ \hline
  max\_depth & 10 & 30 & 40 & 40\\ \hline
  \end{tabular}}
  \label{gs-rf}
\end{table}

\begin{table}[ht]
 \centering
  \caption{SVMのグリットサーチの結果}
  \vspace{0.3\baselineskip}
     \scalebox{0.8}{
  \begin{tabular}{|c|c|c|c|c|} \hline
  パラメタ & 全文 & \shortstack{全文\\(機能語のみ)} & セリフ & \shortstack{セリフ\\(機能語のみ)}\\ \hline \hline
  kernel & sigmoid & sigmoid & poly & rbf\\ \hline
  gamma & 0.001 & 0.001 & 0.01 & 0.0001\\ \hline
  C & 100 & 100 & 1000 & 1000\\ \hline
  \end{tabular}}
  \label{gs-svm}
\end{table}

グリッドサーチの結果を表\ref{gs-rf}表と表\ref{gs-svm}に示す.
それぞれの場合で値が大きく異なるパラーメタが得られる場合があった.


\subsection{評価結果}
\begin{table}[H]
 \centering
  \caption{ランダムフォレストでの結果}
  \vspace{0.3\baselineskip}
  \scalebox{0.9}{
  \begin{tabular}{|l|c|c|c|c|} \hline
    対象 & 正確度 & 適合率 & 再現率 & F値  \\ \hline \hline
    全文 & 0.82 & 0.65 & 0.65 & 0.65 \\ \hline
    全文(機能語) & 0.82 & 0.65 & 0.65 & 0.65\\ \hline
    セリフのみ & 0.68 & 0.36 & 0.36 & 0.36 \\ \hline
    セリフのみ(機能語) & 0.68 & 0.37 & 0.37 & 0.37 \\ \hline
  \end{tabular}
  }
  \label{res-rf}
\end{table}

\begin{table}[H]
 \centering
  \caption{SVMでの結果}
  \vspace{0.3\baselineskip}
  \scalebox{0.9}{
  \begin{tabular}{|l|c|c|c|c|} \hline
    対象 & 正確度 & 適合率 & 再現率 & F値  \\ \hline \hline
    全文 & 0.82 & 0.57 & 064 & 0.59   \\ \hline
    全文(機能語) & 0.80 & 0.36 & 0.60 &  0.45 \\ \hline
    セリフ & 0.70 & 0.38 & 0.22 & 0.22   \\ \hline
    セリフ(機能語) & 0.70 & 0.15 & 0.39 & 0.22   \\ \hline
  \end{tabular}
  }
  \label{res-svm}
\end{table}

\begin{table}[H]
 \centering
  \caption{ナイーブベイズでの結果}
  \vspace{0.3\baselineskip}
  \scalebox{0.9}{
  \begin{tabular}{|l|c|c|c|c|} \hline
    対象 & 正確度 & 適合率 & 再現率 & F値  \\ \hline \hline
    全文 & 0.81 & 0.63 & 0.63 & 0.63 \\ \hline
    全文(機能語) & 0.82 & 0.64 & 0.64 & 0.64 \\  \hline
    セリフ & 0.69 & 0.38 & 0.38 & 0.38 \\ \hline
    セリフ(機能語) & 0.69 & 0.37 & 0.37 & 0.37 \\ \hline
  \end{tabular}
  }
  \label{res-nb}
\end{table}

ランダムフォレストとSVM及びナイーブベイズでの結果を表\ref{res-rf}と表\ref{res-svm}及び\ref{res-nb}に示す.
なお、(機能語)とは学習や推定時に機能語のみを用いた場合を示す.
全体としてF値は高くない結果となった.
特にSVMの場合はSVMでは,全文に機能語を絞らずに分類を行った場合を除く他のすべての場合で,推定が一つの感情に偏ってしまった.
ランダムフォレストとナイーブベイズでは全体としてわずかながらランダムフォレストにの方がよい結果が得られた。
しかし、セリフにおいてはナイーブベイズがランダムフォレストを上回った。

%TODO confusion matrix

\section{考察}
全体としてのF値は高くない結果となった.
原因として学習データが少ないことや出現を示すベクトルの形式(bag-of-words)に問題がある可能性がある.

機械学習の手法を比較する.
SVMでは,全文に機能語を絞らずに分類を行った場合を除く他のすべての場合で,推定が一つの感情に偏ってしまった.
したがってSVMは本研究にふさわしくない可能性が高い。
また、わずかながらナイーブベイズよりランダムフォレストの方が良い結果を示した。
したがって,本研究の目的のためにはSVMよりランダムフォレストの方が有用であると言える.

機能語に絞った場合ととそうでない場合を比較する.
ランダムフォレストの値ではほとんど同じF値を得た。
本実験では学習データに用いた物語が5つと少ないことに加えleave-one-outを用いたことで推定する文と同じ物語の文を用いて学習・推定を行っているため、内容語・機能語の両方を使った推定では内容に依存した語を用いて推定を行っている可能性がある。
したがって、機能語だけでも推定が行える可能を示せたと考えられる。
実際の運用では未知の物語の文に対して推定を行うので,機能語だけの学習・推定の方が精度が高い推定が行えるかもしれない.



\section{本章のまとめ}
本章では,グリッドサーチの結果と各比較実験の結果を示し、それを踏まえて考察を行った。

%全体で評価が確定したものは全体で69.8\%であった.
%全文とセリフのみに絞った場合の比較を行う.
%得られた学習データは表\ref{emotion-count}の通り,Normal以外の感情はセリフに多く含まれることがわかる.
%したがってセリフの感情推定の精度を上げることで全体の精度をあげることができることがわかる.
%全文にくらべセリフのみを対象とした場合はより均等に感情が別れているため分類がより難しい.

