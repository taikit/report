\chapter{結論}

本章では, 本研究のまとめ及び今後の展望について述べる.

\section{まとめ}
本研究では未知の文に対しその文を読み上げるときの感情として最適なものを推測することを目的とした.
このための手法として物語に依存しがちな内容語を除いて機能語のみを用いてランダムフォレストで学習・推定する手法を提案した.


実験はネット上の5つの物語を使用して音声データを作成しWebのアンケートシステムを用いてNormal,Happy,Sad,Angryの4つのに分類し学習データを作成した.
また,比較実験として機能語のみで学習・推定するか否かやランダムフォレストの他にSVMでの実験やセリフ文のみに絞った場合を行った.


結果とした全体的に高い精度を得ることはできなかった.
しかし,本研究にはランダムフォレストが有用であることや内容語を取り去って機能語のみで学習・分類を行っても,精度にほとんど差がないことがわかった.
したがって,ランダムフォレストを用いて,物語を増やし学習データを増やして学習を行い未知の物語に対して推定する検証を行うことで本手法の有用性が証明される可能性がある.

\section{今後の展望}
本研究では物語の数が5つ少なく,またleave-one-outを用いて交差検証を行ったため,内容依存の推定が可能であった可能性がある.
本手法の有効性を正しく検証するには,学習データを増やして学習を行い,まったく未知の物語に対して推定を行うことが望まれる.

また,布目ら\cite{fume}の研究によると,感情推定の精度が多少低くとも感情表現がなされていれば,感情表現がまったくないときに比べて,高い満足度を得られるというユーザー実験の結果もある.
本研究は正しく感情を推定することを目的としているが,あくまで自然な朗読システムの構築が大目的である.
したがって,本研究の結果を用いてユーザー実験を行い,その満足度を確かめる必要があると考える.


