\chapter*{学士論文概要}
\addcontentsline{toc}{chapter}{学士論文概要}

%社会的背景
近年,従来の書籍の他に電子書籍など様々な書籍の楽しみ方が広がっている.
朗読音声を収録したオーディオブックが近年インターネットを介して気軽にダウンロードして楽しめる環境が整ったことなどによりアメリカとカナダの市場規模が急速に拡大している。
日本においても今後さらに普及する可能性がある.
しかし,オーディオブックは書籍から音声化する際にはコストが電子書籍にくらべて10倍ほどかかっており2〜3ヶ月ほどかかると言われている。

%技術的・研究的背景・問題
そこで,電子書籍から人間の音声を人工的に作り出す音声合成技術を用いて機械で自動的に朗読するシステムの研究が行われている.
近年の音声合成技術を用いれば喜怒哀楽といった感情を指定することで感情豊かな音声を合成できる.
しかし,これらのパラメタ人手で設定する必要があり、小説といった膨大な文章に対して都度人手でパラメタ調整を行うのは手間がかかる.

%目的
そこで本研究では未知の文に対しその文を読み上げるときの感情として最適なものを推測することを目的とする.
これにより自然な朗読システムが可能が実現可能になり,オーディオブック作成時のコスト削減が期待される。

%目的を達成するための手法
本研究では,文にどの単語が含まれているかという出現情報をもとに機械学習技術を用いて感情を推定する.
名詞,動詞,形容,形容動詞は内容語とよばれるが物語に依存する可能性が高いため,内容を除いた機能語を用いることで内容に依存しない分類器を生成できる.
このため内容語を無視して機能語のみを用いて学習し感情の推定を行う.
%実験手法の正しさの確認
本研究では,4つに感情をクラス分けをする.
本手法の正しさを確認するために,まず一つの文にそれぞれ4つの感情で音声を合成し、Webアンケートシステムを用いて被験者にこれらの音声を聞いてもらい,どの感情が最も適しているか判定してもらう.
このように生成された学習データを用い交差検証を行うことで本手法の分類性能を評価する.
