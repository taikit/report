\begin{thebibliography}{99}

%TODO 順番確認
% http://ci.nii.ac.jp/naid/110009971766
\bibitem{wsj} Jennifer Maloney,"The Fastest-Growing Format in Publishing: Audiobooks",http://www.wsj.com/articles/the-fastest-growing-format-in-publishing-audiobooks-1469139910,Wall Steet Journal
\bibitem{cnet} 佐藤和也,"高い継続率は「耳がさみしくなるから」--オトバンクに聞くオーディオブック市場と利用動向",https://japan.cnet.com/article/35076656/
\bibitem{ueda} 上田渉,"「耳で聴く読書文化」を築く",http://www.ajec.or.jp/interview\_width\_ueda1/,一般社団法人日本編集制作協会
\bibitem{sugifuji} 杉藤美代子; 大山玄. 朗読におけるポーズと呼吸―息継ぎのあるポーズと息継ぎのないポーズ―. 音声言語. 1990.
\bibitem{yoshimura} YOSHIMURA, Takayoshi. Simultaneous modeling of phonetic and prosodic parameters, and characteristic conversion for HMM-based text-to-speech systems. 2002. PhD Thesis. Nagoya Institute of Technology.
\bibitem{iida} 飯田朱美, and 安村通晃. "感情表現が可能な合成音声の作成と評価." 情報処理学会論文誌 40.2 (1999): 479-486.
\bibitem{tsuduki} 都築亮介, et al. HMM 音声合成における感情表現のモデル化 (合成, 韻律, 生成, 一般). 電子情報通信学会技術研究報告. SP, 音声, 2003, 103.264: 25-30.
\bibitem{habe} 波部斉,”ランダムフォレスト”,情報処理学会研究報告 2012

\bibitem{yoshida} 吉田有里,奥平康弘,田村直良,”音声合成による朗読システムに関する研究”,情報科学技術フォーラム講演論文集,2009:p337-380
\bibitem{otani} 大谷大和, et al. "HMM に基づく感情音声合成のための共有感情付与モデル (オーガナイズドセッション 「文脈や状況に合った発声を実現する音声合成技術及び周辺技術」, 合成, 韻律, 生成, 音声一般)." 電子情報通信学会技術研究報告. SP, 音声 114.303 (2014): 13-18.
\bibitem{fume} 布目光生,鈴木優,森田眞弘,”自然で聞きやすい電子書籍読上げのための文書構造解析技術,東芝レビュー,2011:p32-35
\bibitem{aozora} 青空文庫,http://www.aozora.gr.jp/
%TODO 他の文も
%\bibitem{shirayuki} グリム,菊池寛訳,”白雪姫”,http://www.aozora.gr.jp/cards/001091/files/42308\_17916.html

%ちゃんとした形式に直す
\bibitem{jtalk} 大浦 圭一郎,酒向 慎司, 徳田 恵一,”日本語テキスト音声合成システム Open JTalk”,日本音響学会春季講論集,2010:p343-344
\bibitem{mecab} "MeCab: Yet Another Part-of-Speech and Morphological Analyze",http://taku910.github.io/mecab/
\bibitem{naist} "NAIST Japanese Dictionary",http://naist-jdic.osdn.jp/
\bibitem{mei} "MMDAgent",http://www.mmdagent.jp/


\end{thebibliography}
